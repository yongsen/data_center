% $Header: /Users/joseph/Documents/LaTeX/beamer/solutions/conference-talks/conference-ornate-20min.en.tex,v 90e850259b8b 2007/01/28 20:48:30 tantau $

\documentclass{beamer}

%\usepackage{CJK}
%\usepackage{CJK,CJKnumb}
%\usepackage{ctex}
\usepackage[english]{babel}
\usepackage[utf8]{inputenc}
%\usepackage{setspace}
\usepackage{graphicx}
\usepackage{bm}
\usepackage{textpos}
\usepackage{algpseudocode}
\usepackage{algorithm}
\usepackage{url}
\usepackage{multirow}
\usepackage{threeparttable}
\usepackage{hyperref}
\usepackage{times}
\usepackage[T1]{fontenc}
\usepackage{ulem,tikz}
% Or whatever. Note that the encoding and the font should match. If T1
% does not look nice, try deleting the line with the fontenc.
\usepackage{stfloats}
\usepackage[caption=false,font=scriptsize]{subfig}
%\usepackage[font=scriptsize,labelfont=scriptsize]{caption}

%\setbeameroption{show notes on second screen}
%\setbeamertemplate{note page}[compress]
\usepackage{csquotes}
%\usepackage[style=numeric-comp,autocite=footnote,citetracker=true,maxnames=1,sorting=none,babel=hyphen,hyperref=true,backend=biber]{biblatex}
\usepackage[style=numeric-comp,autocite=footnote,citetracker=true,maxnames=1,sorting=none]{biblatex}
%\usepackage[style=verbose,autocite=footnote,citetracker=true,maxnames=1,sorting=none,babel=hyphen,hyperref=true,backend=biber]{biblatex}

\usetikzlibrary{arrows}
\tikzstyle{block}=[draw opacity=0.7,line width=1.4cm]
\setbeamertemplate{caption}[numbered]
\captionsetup{font=scriptsize,labelfont=scriptsize}
\setbeamerfont{caption}{size=\scriptsize}
\hypersetup{pdfpagemode=FullScreen}
\setbeamercovered{invisible}
\setlength{\skip\footins}{25cm plus 0cm minus 25cm}
\setlength{\footnotesep}{0.3cm}
\renewcommand*{\bibfont}{\tiny}

\mode<presentation>
{%\usetheme{Boadilla}
%  %\usetheme{Copenhagen}
%  \setbeamertemplate{navigation symbols}{}
  \usetheme{Madrid}
  \setbeamercovered{transparent}
  \setbeamertemplate{sidebar right}{}% or get rid of navigation entries there somehow
  %\addtobeamertemplate{footline}{\hfill\usebeamertemplate***{navigation symbols}\par}{}
}


\DeclareCiteCommand{\footfullcitetext}
  [\let\thefootnote\relax\mkbibfootnotetext]
  {\usebibmacro{prenote}}
  {\mkbibbrackets{\thefield{labelnumber}}%
   \addnbspace
   \usedriver
     {\DeclareNameAlias{sortname}{default}}
     {\thefield{entrytype}}}
  {\multicitedelim}
  {\usebibmacro{postnote}}

\makeatletter

\let\cbx@citehook=\empty
\newtoggle{cbx@blockcite}

\renewcommand{\@makefntext}[1]{%
  \noindent\normalfont\@thefnmark#1}

\DeclareCiteCommand{\sfcite}[\cbx@superscript]%
  {\usebibmacro{cite:init}%
   \let\multicitedelim=\supercitedelim
   \iffieldundef{prenote}{}{\BibliographyWarning{Ignoring prenote argument}}%
   \iffieldundef{postnote}{}{\BibliographyWarning{Ignoring postnote argument}}}
  {\usebibmacro{citeindex}%
   \ifciteseen
     {\ifnumequal{\value{page}}{\csuse{cbx@page@\thefield{entrykey}}}
       {}
       {\ifnumequal{\value{framenumber}}{\csuse{cbx@frame@\thefield{entrykey}}}
          {\usebibmacro{sfcite}}
          {}}}
     {\usebibmacro{sfcite}}%
   \usebibmacro{cite:comp}}
  {}
  {\usebibmacro{cite:dump}}

\newbibmacro*{sfcite}{%
  \csnumgdef{cbx@page@\thefield{entrykey}}{\value{page}}%
  \csnumgdef{cbx@frame@\thefield{entrykey}}{\value{framenumber}}%
  \xappto\cbx@citehook{%
    \noexpand\footfullcitetext{\thefield{entrykey}}}}

\newrobustcmd*{\cbx@superscript}[1]{%
  \mkbibsuperscript{\mkbibbrackets{#1}}%
  \iftoggle{cbx@blockcite}
    {}
    {\cbx@citehook%
     \global\let\cbx@citehook=\empty}}

\BeforeBeginEnvironment{block}{\global\toggletrue{cbx@blockcite}}

\def\metabox#1{\edef\theprevdepth{\the\prevdepth}\nointerlineskip
  \vbox to0pt{#1\vss}\prevdepth=\theprevdepth}

\AfterEndEnvironment{block}
  {\metabox{%
     \global\togglefalse{cbx@blockcite}%
     \cbx@citehook%
     \global\let\cbx@citehook=\empty}}


\BeforeBeginEnvironment{exampleblock}{\global\toggletrue{cbx@blockcite}}

\def\metabox#1{\edef\theprevdepth{\the\prevdepth}\nointerlineskip
  \vbox to0pt{#1\vss}\prevdepth=\theprevdepth}

\AfterEndEnvironment{exampleblock}
  {\metabox{%
     \global\togglefalse{cbx@blockcite}%
     \cbx@citehook%
     \global\let\cbx@citehook=\empty}}

\AtEveryCitekey{\iffootnote{\tiny}{\color{blue}}{\vspace{-1ex}}}
\renewcommand*{\footnoterule}{\kern -1pt \hrule \@width 2in \kern 1pt}

\makeatother

%\addtobeamertemplate{footnote}{\vspace{-6pt}\advance\hsize-0.5cm}{\vspace{6pt}}

%\setbeamersize{text margin left=8pt,text margin right=8pt}
\addbibresource{Bib.bib}

%% Code for placing the footnote above the navigiation symbols
%\addtobeamertemplate{footnote}{\vspace{-6pt}}{\vspace{6pt}}
%\makeatletter
%% Alternative A: footnote rule
%\renewcommand*{\footnoterule}{\kern -3pt \hrule \@width 2in \kern 8.6pt}
%% Alternative B: no footnote rule
%% \renewcommand*{\footnoterule}{\kern 6pt}
%\makeatother

\title[Multipath TCP in Optical Data Centers] % (optional, use only with long paper titles)
{Multipath TCP in Optical Data Centers}

%\subtitle
%{�о���״}

\author[Yongsen Ma] % (optional, use only with lots of authors)
{Yongsen Ma}
%{F.~Author\inst{1} \and S.~Another\inst{2}}
% - Give the names in the same order as the appear in the paper.
% - Use the \inst{?} command only if the authors have different
%   affiliation.

\institute[SJTU] % (optional, but mostly needed)
{ %�Ϻ���ͨ��ѧ
  \begin{figure}
  \includegraphics[width=1.2in]{SJTU.pdf}
  \end{figure}
}
% - Use the \inst command only if there are several affiliations.
% - Keep it simple, no one is interested in your street address.

\date[\today] % (optional, should be abbreviation of conference name)
{\today}
% - Either use conference name or its abbreviation.
% - Not really informative to the audience, more for people (including
%   yourself) who are reading the slides online

\subject{Experimental Computer Science}
% This is only inserted into the PDF information catalog. Can be left
% out.

% \pgfdeclareimage[height=5cm]{SJTU}{SJTUW.pdf}
% \logo{\pgfuseimage{SJTU}}

% Delete this, if you do not want the table of contents to pop up at
% the beginning of each subsection:
\AtBeginSubsection[]
{
  \begin{frame}<beamer>{Outline}
    \tableofcontents[currentsection,currentsubsection]
  \end{frame}
}

% If you wish to uncover everything in a step-wise fashion, uncomment
% the following command:
%\beamerdefaultoverlayspecification{<+->}

\begin{document}
%\begin{CJK*}{GBK}{hei}
%\renewcommand{\today}{\CJKnumber{\the\month}��}
%\renewcommand\figurename{ͼ}

%\addtobeamertemplate{frametitle}{}{%
%\begin{textblock*}{100mm}(0.9\textwidth,-0.5cm)
%\includegraphics[height=0.8cm]{SJTUB.pdf}
%\end{textblock*}
%}

%\addtobeamertemplate{frametitle}{}{%
%\begin{textblock*}{100mm}(0.8\textwidth,-0.8cm)
%\includegraphics[height=0.7cm]{SJTU.pdf}
%\end{textblock*}
%}

%\addtobeamertemplate{frame}{}{%
%\begin{tikzpicture}[remember picture,overlay]
%\node[anchor=north east] at (current page.north east) {\includegraphics[height=0.9cm]{SJTUW.pdf}};
%\end{tikzpicture}}

%\begin{frame}
%  \titlepage
%\end{frame}
%
%\begin{frame}{Outline}
%  \tableofcontents
%  % You might wish to add the option [pausesections]
%\end{frame}

\begin{frame}{Motivation}
\begin{block}{Multipath TCP and Optical Switching in Data Centers}
Multipath TCP and optical switching are respectively explored recently. Both are emerging technologies and represent the future trend.
\begin{itemize}
  \item Multipath TCP provides tradeoff between reliability and utilization. But it requires additional modification of architecture and topology.
  \item Optical Switching allows flexible strategies on topology control and demand response, which can get higher bandwidth and efficiency. But its reconfiguration delays will lead to high latency.
\end{itemize}
Benefits when multipath TCP and optical switching are combined:
\begin{itemize}
  \item Optical switching enables multiplexing technology which provides preconditions of architecture and topology for multipath TCP.
  \item Multipath TCP has lower flow completion time which reduces the adverse effect of high latency in optical switching.
\end{itemize}
\end{block}
\end{frame}

\begin{frame}{Challenges}
In spite of all the benefits, both multipath TCP and optical switching will make it more complex and challenging for data center networking.

\begin{block}{Topology: relay nodes, extra overhead, efficient throughput}
\begin{itemize}
  \item BCube and DCell need relay servers to support multipath, but it makes capacity scheduling more complicated.
  \item In Fattree and VL2 that do not need relay nodes, multipath leads to extra overhead such as SYN and coding headers.
  \item The extra overhead leads to the decrease of efficient throughput, though the flow completion time is reduced.
\end{itemize}

\end{block}

\begin{block}{Failures Handling: performance, robustness, reliability}
\begin{itemize}
  \item When failures occur, the performance and robustness are reduced considering SYN\&coding headers and flow drops\&retransmission.
  \item Failures in nodes (including servers and switches) and links result in poor reliability compared to single path forwarding.
\end{itemize}
\end{block}
\end{frame}

\begin{frame}{Challenges}
\begin{block}{Congestion Control}
\begin{itemize}
  \item Apart from node and link failures, network congestion occurs more frequently in multipath, especially when the traffic loads are heavy.
\end{itemize}
\end{block}
\begin{block}{Resource Allocation}
\begin{itemize}
  \item Both multipath and flexible switching make it more complicated for data center networking, like addressing, routing, scheduling, etc.
\end{itemize}
\end{block}
Therefore, link failures (including network congestion) of multipath TCP have significant influence on the reliability and efficiency of optical data centers. The followings can be introduced to address these problems.
\begin{itemize}
  \item Flow Backup: encoding, synchronization, retransmission
  \item Block ACK: congestion control, failure detection, state feedback
\end{itemize}
\end{frame}

\begin{frame}{Multipath TCP in Data Centers}{Trade-off between reliability and utilization}
\begin{figure}[!t]
\centering
    \includegraphics[width=0.8\textwidth]{multipath1.pdf}
\end{figure}
\end{frame}

\begin{frame}{Multipath TCP in Data Centers}{Low reliability due to drops and retransmission}
\begin{figure}[!t]
\centering
    \includegraphics[width=0.8\textwidth]{multipath2.pdf}
\end{figure}
\end{frame}

\begin{frame}{Failures Handling in Data Centers}{High reliability through flows backup}
\begin{figure}[!t]
\centering
    \includegraphics[width=0.8\textwidth]{multipath3.pdf}
\end{figure}
\end{frame}

\begin{frame}{Failures Handling in Data Centers}{High efficiency through flexible switching}
\begin{figure}[!t]
\centering
    \includegraphics[width=0.8\textwidth]{multipath4.pdf}
\end{figure}
\end{frame}

\begin{frame}{Multipath TCP and Failures Handling}{Basic Procedures}
\begin{block}{Topology \& Congestion Control, Packet Scheduling \& Decoding}
\begin{enumerate}
  \item Split flow(s) into subflows and push into backup pool
  \item Congestion control by length of \textbf{flow backup} or \textbf{block ACK}
  \item Topology control and scheduling by \textbf{flexible switching}
  \item Add premix to subflows, make backup and transmit subflows
  \item Decoding according to premix or flow backup
\end{enumerate}
\end{block}
\begin{itemize}
  \item Flow backup: copy the transmitting subflows and premix
  \item Block ACK: several backup subflows use one combined ACK
  \item Flexible switching: on-demand response by block ACK and backup
\end{itemize}
\end{frame}

\begin{frame}{Multipath TCP and Failures Handling}{Adopted Technologies}
\begin{block}{Failures handling technologies}
\begin{itemize}
  \item \textbf{Flow backup} reduces the overhead of drops and retransmission: T(2,2) $\rightarrow$ T(1,1).
  \item \textbf{Flexible switching} improves the capacity and reduces the delay: 10G $\rightarrow$ 20G.
\end{itemize}
\end{block}
The above failures handling methods improve the decoding efficiency: after transmit $\rightarrow$ as transmitting.
\begin{block}{Multipath TCP technologies}
\begin{itemize}
  \item \textbf{Block ACK} reduces extra overhead: ACK/subflow $\rightarrow$ ACK/block, and gives state feedback for failure detection and path selection.
\end{itemize}
\end{block}
The multipath routing and scheduling of optical data centers have not been explored, especially on flexible switching (topology control) and failures handling (including congestion control).
\end{frame}
\end{document}


